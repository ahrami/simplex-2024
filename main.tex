\documentclass[a4paper,article,14pt]{extarticle}
\usepackage{styles}
\usepackage{amsmath}

\let\phi = \varphi
\let\epsilon = \varepsilon

\title{Симплекс Метод}

\begin{document}

\maketitle
\newpage
\tableofcontents
\newpage

\section{Линейное программирование}

Задача линейного программирования (ЛП) состоит в том, что нам необходимо максимизировать или минимизировать некоторый линейный функционал на многомерном пространстве при заданных линейных ограничениях.

\subsection{Линейный функционал}

Линейный функционал еще называется линейной формой, 1-формой, ковектором и ковариантным вектором.

Линейный функционал это линейное отображение, действующее из векторного пространства над полем в это же поле.

В алгебре и геометрии обычно используют название линейная форма, потому что чаще идет речь о конечномерных векторных пространствах, а в функциональном анализе линейный функционал, потому что там часто отображение именно над множествами функций.

Рациональные или вещественные множества это примеры полей.
Элементы поля называются скалярами.

Векторное пространство так же называется линейным пространством или линеалом.
Векторное пространство задается над полем.

Погожев определял линеал как множество объектов произвольной природы для которых определены сложение и умножение на число.
Никаких полей у нас не вводилось.

Отображение называется линейным когда оно удовлетворяет двум свойствам линейности:
\begin{align}
    f(a + b) & = f(a) + f(b) \\
    f(\lambda a) & = \lambda f(a)
\end{align}

\subsection{Геометрическое представление задачи ЛП}

Симплекс --- это многомерное обобщение треугольника.
Другое его название --- n-мерный тетраэдр.
Симплекс --- это выпуклая оболочка \(n+1\) точек афинного пространства размерности как минимум \(n\), которые не лежат в подпространстве размерности \(n-1\) (афинно независимы).

Вообще говоря каждое из линейных неравенств на переменные ограничивает полупространство в соответствующем линеале.
В результате все неравенства ограничивают выпуклый многогранник.

Линейный функционал порождает гиперплоскость.
Гиперплоскость это подпространство, размерность которого на 1 меньше исходного пространства.

Требуется найти такую гиперплоскость, чтобы значение функционала было максимальным (или минимальным) и чтобы гиперплоскость пересекала многогранник хотя бы в одной точке.

Гиперплоскость задается одним вектором и этим вектором будет вектор \(n = \frac c {|c|}\). Это вектор самого быстрого изменения целевой функции и называется градиентом.
Еще для задания гиперплоскости нужна координата или же точка.

\subsection{Обозначения для математического представления}

\begin{itemize}
    \item Целевая функция (ЦФ) обозначается \(z\).
    \item Вектор коэффициентов целевой функции будем обозначать через \(c\).
    \item Вектор переменных \(x\).
    \item Вектор переменных двойственной задачи \(y\).
\end{itemize}

Вещи относящиеся к решению будем отмечать звездочкой:

\begin{itemize}
    \item Оптимальное решение: \(x^*\), \(y^*\).
    \item Значение задачи: \(z^*\), \(\overline z^*\)
\end{itemize}

Векторы у нас вертикальные, запись в строчку использует транспонирование.

\subsection{Математическое описание задачи ЛП}

Прямая задача максимизации в стандартной форме:
\begin{equation} \label{eq:primal_max}
    \begin{aligned}
        \max z & = \max c^Tx \\
        Ax & \le b \\
        x & \ge 0
    \end{aligned}
\end{equation}

% Можно писать еще так, но мы не будем:
% \begin{equation}
%     z = \langle c,x \rangle
%     , \quad
%     z \rightarrow \max
% \end{equation}

У нас \(m\) ограничений и \(n\) переменных.
\begin{equation}
    \begin{gathered}
        \begin{aligned}
            x & = (x_1, x_2, \ldots, x_n)^T \\
            c & = (c_1, c_2, \ldots, c_n)^T \\
            b & = (b_1, b_2, \ldots, b_m)^T
        \end{aligned} \\
        A
        = \{a_{ij}\}_{i,j = 1}^{m,n}
        =
        \begin{pmatrix}
            a_{11} & a_{12} & \ldots & a_{1n} \\
            a_{21} & a_{22} & \ldots & a_{2n} \\
            \vdots & \vdots & \ddots & \vdots \\
            a_{m1} & a_{m2} & \ldots & a_{mn}
        \end{pmatrix}
        % = \begin{bmatrix}a_1 \\ a_2 \\ \vdots \\ a_n\end{bmatrix}
        % = [a^1, a^2, \ldots, a^n]
    \end{gathered}
\end{equation}

Другая раскрытая запись задачи ЛП для того чтобы думать:
\begin{equation}
    \begin{gathered}
        \max z = \max (c_1x_1 + c_2x_2 + \ldots + c_nx_n) \\
        \begin{pmatrix}
            a_{11} & a_{12} & \ldots & a_{1n} \\
            a_{21} & a_{22} & \ldots & a_{2n} \\
            \vdots & \vdots & \ddots & \vdots \\
            a_{m1} & a_{m2} & \ldots & a_{mn}
        \end{pmatrix}
        \begin{pmatrix}
            x_1 \\ x_2 \\ \vdots \\ x_n
        \end{pmatrix}
        \le
        \begin{pmatrix}
            b_1 \\ b_2 \\ \vdots \\ b_m
        \end{pmatrix}
        \\
        (x_1, x_2, \ldots, x_n)^T \ge 0
    \end{gathered}
\end{equation}

\section{Двойственность}

Если честно, я бы дал это в самом начале, как такой разгон перед симплексом.
То есть если мы даем теорию ЛП, то и двойственность сразу надо, а потом уже алгоритмы решения.

\subsection{Двойственная задача}

В английском используются термины primal и dual.
На русском будем говорить прямая и двойственная.

Двойственная задача для стандартной задачи максимизации (\ref{eq:primal_max}):
\begin{equation} \label{dual_max}
    \begin{aligned}
        \min \overline z & = \min b^Ty \\
        A^Ty & \ge c \\
        y & \ge 0
    \end{aligned}
\end{equation}

В двойственной задаче у нас наоборот \(m\) переменных и \(n\) ограничений
\begin{equation}
    \begin{gathered}
        y = (y_1, y_2, \ldots, y_m)^T \\
        A^T =
        \begin{pmatrix}
            a_{11} & a_{21} & \ldots & a_{m1} \\
            a_{12} & a_{22} & \ldots & a_{m2} \\
            \vdots & \vdots & \ddots & \vdots \\
            a_{1n} & a_{2n} & \ldots & a_{mn}
        \end{pmatrix}
    \end{gathered}
\end{equation}

Более раскрытая запись для того чтобы думать:
\begin{equation}
    \begin{gathered}
        \min \overline z = \min (b_1y_1 + b_2y_2 + \ldots + b_my_m) \\
        \begin{pmatrix}
            a_{11} & a_{21} & \ldots & a_{m1} \\
            a_{12} & a_{22} & \ldots & a_{m2} \\
            \vdots & \vdots & \ddots & \vdots \\
            a_{1n} & a_{2n} & \ldots & a_{mn}
        \end{pmatrix}
        \begin{pmatrix}
            y_1 \\ y_2 \\ \vdots \\ y_m
        \end{pmatrix}
        \ge
        \begin{pmatrix}
            c_1 \\ c_2 \\ \vdots \\ c_n
        \end{pmatrix}
        \\
        (y_1, y_2, \ldots, y_m)^T \ge 0
    \end{gathered}
\end{equation}

\subsection{Теоремы двойственности}

\begin{equation}
    \begin{gathered}
        c^T x \le b^T y \\
        c^T x^* = b^T y^*
    \end{gathered}
\end{equation}

\(x*\) называется оптимальным решением, а значение целевой функции при нем называется значением задачи.

Вообще говоря, приведение ограничений к равенствам дает нам задачу ЛП в канонической форме.

\subsection{Двойственность задачи ЛП в общем, стандартном и каноническом виде}

Прямая задача в общем виде:
\begin{equation}
    \begin{gathered}
        \max z = \max c^T x \\
        \begin{aligned}
            a_ix & \le b_i, \quad i = \overline{1, m_1} \\
            a_ix & = b_i, \quad i = \overline{m_1 + 1, m} \\
            x_j & \ge 0, \quad j = \overline{1, n_1} \\
            x_j & \in R, \quad j = \overline{n_1 + 1, n} \\
        \end{aligned}
    \end{gathered}
\end{equation}

Двойственная к ней:
\begin{equation}
    \begin{gathered}
    \min \overline z = \min b^T y \\
        \begin{aligned}
            (a^j)^Ty & \ge c_j, \quad j = \overline{1, n_1} \\
            (a^j)^Ty & = c_j, \quad j = \overline{n_1 + 1, n} \\
            y_i & \ge 0, \quad i = \overline{1, m_1} \\
            y_i & \in R, \quad i = \overline{m_1 + 1, m} \\
        \end{aligned}
    \end{gathered}
\end{equation}

Частный случай прямой задачи при котором она находится в стандартном виде означает что \(m_1 = m\) и \(n_1 = n\).
При таких значениях все ограничения являются неравенствами и у всех переменных есть ограничение на знак.
\begin{equation}
    \begin{gathered}
        \max z = \max c^T x \\
        \begin{aligned}
            a_ix & \le b_i, \quad i = \overline{1, m} \\
            x_j & \ge 0, \quad j = \overline{1, n} \\
        \end{aligned}
    \end{gathered}
\end{equation}

Двойственная задача к этой стандартной задаче будет тоже иметь все ограничения в виде неравенств и будет иметь ограничения на знак для переменных.
\begin{equation}
    \begin{gathered}
    \min \overline z = \min b^T y \\
        \begin{aligned}
            (a^j)^Ty & \ge c_j, \quad j = \overline{1, n} \\
            y_i & \ge 0, \quad i = \overline{1, m} \\
        \end{aligned}
    \end{gathered}
\end{equation}

Частный случай прямой задачи при котором она находится в каноническом виде означает что \(m_1 = 0\) и \(n_1 = n\).
При таких значениях все ограничения являются равенствами и у всех переменных есть ограничение на знак.
\begin{equation}
    \begin{gathered}
        \max z = \max c^T x \\
        \begin{aligned}
            a_ix & = b_i, \quad i = \overline{1, m} \\
            x_j & \ge 0, \quad j = \overline{1, n} \\
        \end{aligned}
    \end{gathered}
\end{equation}

Двойственная задача к этой канонической задаче будет иметь все ограничения в виде неравенств и не будет иметь ограничений на знак для переменных.
\begin{equation}
    \begin{gathered}
    \min \overline z = \min b^T y \\
        \begin{aligned}
            (a^j)^Ty & \ge c_j, \quad j = \overline{1, n} \\
            y_i & \in R, \quad i = \overline{1, m} \\
        \end{aligned}
    \end{gathered}
\end{equation}

\subsection{Представление двойственной задачи. Оптимальное решение двойственной задачи.}

Что такое вообще представление двойственной задачи.
Типа математическое?

Оптимальное решение двойственной задачи это типа оно выводится з прямой безо всякого алгоритма?
если так то было бы круто и если эта вся теория не зависит от симплекс метода то ее можно всю дать либо в самом начале перед симплекс методом, либо дать очень обзорно хоть вначале хоть в конце, раз у нас фокус то все таки на симплекс методе а не на ЛП в целом.

Я уверен ПМИшники про двойственность уже всё знают и нет особого смысла застревать на двойственноти.
Но если тут есть какие нибудь супер важные доказательства то мб и стоит их показать, хоть и не подробно.

Посмотрим сколько будет получаться по симплексу, и если будем чувствовать что успеваем то добавим в рассказа и про двойственность.

Все равно странно что нас просят двойственность в симплекс методе.

\section{Основы симплекс метода}

Это метод решения задачи ЛП.

Он истекает из графического метода решения и из факта что оптимальное решение будет находиться в краевой точке множества которое задано нашими ограничениями.
Краевая точка это ..
И почему вообще так.

Симплекс метод - алгоритм решения оптимизационной задачи линейного программирования путем перебора вершин выпуклого многогранника в многомерном пространстве.

Сущность метода: построение базисных решений, на которых монотонно убывает линейный функционал до ситуации, когда выполняются необходимые условия локальной оптимальности.

В работе Канторовича 1939 впервые были изложены принципы отрасли которую потом назвали линейным программированием.

\subsection{Принцип симплекс метода}

Принцип симплекс метода состоит в том что мы выбираем одну вершину многогранника и перемещаемся по ребрам в другие вершины в сторону увеличения функционала.

Когда такой переход невозможен считается что оптимальное значение найдено.

Симплекс метода можно поделить на две основные фазы:

\begin{enumerate}
    \item Нахождение исходной вершины множества допустимых решений,
    \item Последовательный переход от одной вершины к другой, ведущий к оптимизации значения целевой функции.
\end{enumerate}

Из за того что в некоторых случаях нахождение исходной вершины тривиально, симплекс метод бывает однофазным и двухфазным.
То есть в тривиальном случае первую фазу опускаем.
Тривиальным случаем может быть когда 0 - допустимое решение

\subsection{Симплекс таблицы}

Симплексная таблица (СТ) - основной элемент вычислительной процедуры симплекс метода.
Симплексная таблица представляет собой таблицу коэффициентов диагоняльной формы, построенной для канонической задачи максимизации.
Из за того что она диагональная, она соответствует базисному решению рассматриваемой системы линейных уравнений.

Симплексная таблица называется прямо-допустимой если \(b \ge 0\).
Симплексная таблица называется двойственно допустимой если \(c \ge 0\).
Симплексная таблица называется оптимальной если она одновременно и прямо-допустимая и двойственно допустимая.
Оптимальная СТ соответствует оптимальному базисному решению.

Алгоритм начинается с прямо-допустимой симплексной таблицы.

На первом шаге мы выбираем ведущий столбец.
Выбираем столбец с минимальным отрицательным \(c\).
Если таких нет, то оптимальное решение найдено.

На втором шаге выбираем ведущую строку из строк у которых элемент этого столбца положительный.
Если положительных нет, задача ЛП не омеет оптимального решения.
Выбираем строку с минимальным \(b/a\).

На третьем шаге превращаем ведущий элемент в 1 и остальные элементы столбца в 0 эквивалентными преобразованиями.
То есть мы проводим процедуру Гаусса по приведению таблицы к диагональному виду по новому базису.

\section{Базисные решения}

Нам нужно показать:
\begin{itemize}
    \item Что такое базисное решение
    \item Как оно выводится, записывается и определяется математически
    \item Почему это именно вершины симплекса
    \item Как искать базисные решения
    \item Как искать именно допустимые базисные решения
\end{itemize}

Если у системы линейных уравнений существует решение, у нее существует и базисное решение.

Базисным решениям соответствуют крайние точки множества допустимых решений.
Небазисные допустимые решения являются внутренними точками множества допустимых решений.

Если задача ЛП имеет допустимое решение, то она имеет и допустимое базисное решение.

Если задача ЛП имеет оптимальное решение, то она имеет и оптимальное базисное решение.
В силу этого утверждения симплекс метод оперирует только с базисными решениями.

Удобно не забывать что мы работаем с пространством исходных переменных, и в него не входят дополнительные.

Добавлением переменной к каждому неравенству при приведении стандартной задачи мы превращаем систему неравенств в линейно независимую систему уравнений.
С другой стороны, Гаусс нам тоже дает подмножество уравнений которые линейно независимы.

На каждом шаге мы разбиваем множество переменных на базисные и небазисные.
Небазисных переменных у нас столько, сколько размерность подпространства решений.
Имеется в виду опять подпространство исходного пространства, то есть без дополнительных переменных.
А подпространство именно потому что в общей задаче могут быть уравнения.

Базисных переменных у нас соответственно столько, сколько у нас линейно независимых уравнений, и каждая базисная переменная соответствует отдельному линейно независимому уравнению.

Каждая дополнительная переменная пропорциональна расстоянию от решения до гиперплоскости соответствующего ограничения.
Это означает, что приравнивание дополнительной переменной к нулю будет гарантировать что мы будем касаться этого ограничения.
Приравнивание к нулю \(n\) таких переменных для линейно независимых ограничений гарантирует что мы будем в вершине множества допустимых решений.

Если же мы приравниваем к нулю одну из исходных переменных, тогда дополнительная в этом неравенстве должна быть положительной, что значит мы не касаемся соответствующего ограничения, но не стоит забывать про ограничения на знак, которого мы как раз и касаемся в этом случае. (В случаях когда исходное ограничение равенство или коэффициент при иксе нулевой у нас просто либо размерность меньше либо ограничения линейно зависимы, поэтому ничего не ломается, я думаю).

Таким образом при обнулении набазисных переменных мы всегда касаемся \(n\) линейно независимых ограничений и оказываемся в какой-нибудь вершине множества допустимых решений.

\section{Матричное представление симплекс таблиц}

Симплекс таблица это способ решения задачи ЛП и способ записи решения симплекс метода.

Стандартную задачу максимизации можно переписать в каноническом виде таким образом:

\begin{equation}
    \begin{gathered}
        \begin{bmatrix}
            1 & -c^T & 0 \\
            0 & A & E_m
        \end{bmatrix}
        \begin{bmatrix}
            z \\ x \\ s
        \end{bmatrix}
        =
        \begin{bmatrix}
            0 \\ b
        \end{bmatrix}
        \\
        z \rightarrow \max, \quad x \ge 0, \quad s \ge 0
    \end{gathered}
\end{equation}

Это и есть симплекс таблица в матричной форме.

\begin{itemize}
    \item \(s \ge 0\) --- новые переменные, дополняющие старые таким образом, что неравенство переходит в равенство.
    \item \(z\) --- это переменная которую необходимо максимизировать, значение нашей целевой функции.
    \item
        Еще в википедии есть что \(b \ge 0\).
        Это как раз условие того что симплекс алгоритм должен начинаться с прямо-допустимой таблицы.
\end{itemize}

Более раскрытая форма для понимания:

\setcounter{MaxMatrixCols}{20}
\begin{equation}
    \begin{pmatrix}
        1 & \vline & -c_1 & -c_2 & \ldots & -c_n & \vline & 0 & 0 & \ldots & 0 \\
        \hline
        0 & \vline & a_{11} & a_{12} & \ldots & a_{1n} & \vline & 1 & 0 & \ldots & 0 \\
        0 & \vline & a_{21} & a_{22} & \ldots & a_{2n} & \vline & 0 & 1 & \ldots & 0 \\
        \vdots & \vline & \vdots & \vdots & \ddots & \vdots & \vline & \vdots & \vdots & \ddots & \vdots \\
        0 & \vline & a_{m1} & a_{m2} & \ldots & a_{mn} & \vline & 0 & 0 & \ldots & 1 \\
    \end{pmatrix}
    \begin{pmatrix}
        z \\ \hline x_1 \\ x_2 \\ \vdots \\ x_n \\ \hline s_1 \\ s_2 \\ \vdots \\ s_m
    \end{pmatrix}
    =
    \begin{pmatrix}
        0 \\ \hline b_1 \\ b_2 \\ \vdots \\ b_m
    \end{pmatrix}
\end{equation}

Нужно написать так же матрично преобразования этих симплекс таблиц.

\section{Условия оптимальности и допустимости}

Наверное это как раз про прямо-допустимые симплекс таблицы и про двойственно-допустимые симплекс таблицы.

\section{Алгоритм решения с ограниченными переменными}

Я не понимаю что это значит...

Целочисленный что ли?

\end{document}
