\documentclass[a4paper,article,14pt]{extarticle}
\usepackage{styles}
\usepackage{amsmath}

\let\phi = \varphi
\let\epsilon = \varepsilon

\title{Симплекс Метод}
\author{Радькова Ирина Тимофеевна, Храмцов Андрей Игоревич}

\begin{document}

% Временное удаление foot на titlepage
\newgeometry{left=30mm, top=20mm, right=15mm, bottom=20mm, nohead, nofoot}
\begin{titlepage}
\begin{center}

\textbf{Санкт--Петербургский}
\textbf{государственный университет}

\vspace{35mm}

\textbf{\textit{\large Храмцов Андрей Игоревич}} \\
\textbf{\textit{\large Радькова Ирина Тимофеевна}} \\[8mm]
% Название
\textbf{\textit{\large Симплекс Метод}}

\vspace{20mm}

% Научный руководитель, рецензент
\begin{flushright}
\begin{minipage}[t]{0.65\textwidth}

\end{minipage}
\end{flushright}

\vfill 

{Санкт-Петербург}
\par{\the\year{} г.}
\end{center}
\end{titlepage}
% Возвращаем настройки geometry обратно (то, что объявлено в преамбуле)
\restoregeometry
% Добавляем 1 к счетчику страниц ПОСЛЕ titlepage, чтобы исключить 
% влияние titlepage environment
\addtocounter{page}{1}

% \maketitle
\newpage
\tableofcontents
\newpage

\section{Линейное программирование}

Задача линейного программирования (ЛП) состоит в том, что нам необходимо максимизировать или минимизировать некоторый линейный функционал на многомерном пространстве при заданных линейных ограничениях.

\subsection{Линейный функционал}

Линейный функционал еще называется линейной формой, 1-формой, ковектором и ковариантным вектором.

Линейный функционал это линейное отображение, действующее из векторного пространства над полем в это же поле.

В алгебре и геометрии обычно используют название линейная форма, потому что чаще идет речь о конечномерных векторных пространствах, а в функциональном анализе линейный функционал, потому что там часто отображение именно над множествами функций.

Рациональные или вещественные множества это примеры полей.
Элементы поля называются скалярами.

Векторное пространство так же называется линейным пространством или линеалом.
Векторное пространство задается над полем.

Погожев определял линеал как множество объектов произвольной природы для которых определены сложение и умножение на число.
Никаких полей у нас не вводилось.

Отображение называется линейным когда оно удовлетворяет двум свойствам линейности:
\begin{equation}
    \begin{aligned}
        f(a + b) & = f(a) + f(b) \\
        f(\lambda a) & = \lambda f(a)
    \end{aligned}
\end{equation}

\subsection{Геометрическое представление задачи ЛП}

Симплекс --- это многомерное обобщение треугольника.
Другое его название --- n-мерный тетраэдр.
Симплекс --- это выпуклая оболочка \(n+1\) точек афинного пространства размерности как минимум \(n\), которые не лежат в подпространстве размерности \(n-1\) (афинно независимы).

Вообще говоря каждое из линейных неравенств на переменные ограничивает полупространство в соответствующем линеале.
В результате все неравенства ограничивают выпуклый многогранник.

Линейный функционал порождает гиперплоскость.
Гиперплоскость это подпространство, размерность которого на 1 меньше исходного пространства.

Требуется найти такую гиперплоскость, чтобы значение функционала было максимальным (или минимальным) и чтобы гиперплоскость пересекала многогранник хотя бы в одной точке.

Уравнение гиперплоскости можно записать так:
\begin{equation}
    \{x | nx = \bar x\}
\end{equation}
где \(n\) это нормаль гиперплоскости, а \(\bar x\) это точка через которую проходит гиперплоскость.

Уравнение полупространства можно записать так:
\begin{equation}
    \{x | nx \ge \bar x\}
\end{equation}

Таким образом каждое ограничение порождает гиперплоскость.

Вектор самого быстрого изменения целевой функции и называется градиентом:
\begin{equation}
n = \frac c {|c|}
\end{equation}

Этот вектор тоже может задать гиперплоскость проходящую через какую-нибудь точку.
Другими словами, он задает совокупность, или семейство гиперплоскостей.

Опорной гиперплоскостью выпуклого множества называется такая гиперплоскость которая содержит по крайней мере одну точку этого множества и все точки данного множества расположены в одном из полупространств, порождаемых гиперплоскостью.

\subsection{Обозначения для математического представления}

\begin{itemize}
    \item Целевая функция (ЦФ) обозначается \(z\).
    \item Вектор коэффициентов целевой функции будем обозначать через \(c\).
    \item Вектор переменных \(x\).
    \item Вектор переменных двойственной задачи \(y\).
    \item слабые переменные \(s\).
\end{itemize}

Вещи относящиеся к решению будем отмечать звездочкой:

\begin{itemize}
    \item Оптимальное решение: \(x^*\), \(y^*\).
    \item Значение задачи: \(z^*\), \(\overline z^*\)
\end{itemize}

Векторы у нас вертикальные, запись в строчку использует транспонирование.

\subsection{Математическое описание задачи ЛП}

Прямая задача максимизации в стандартной форме:
\begin{equation} \label{eq:primal_max}
    \begin{aligned}
        \max z & = \max c^Tx \\
        Ax & \le b \\
        x & \ge 0
    \end{aligned}
\end{equation}

% Можно писать еще так, но мы не будем:
% \begin{equation}
%     z = \langle c,x \rangle
%     , \quad
%     z \rightarrow \max
% \end{equation}

У нас \(m\) ограничений и \(n\) переменных.
\begin{equation}
    \begin{gathered}
        \begin{aligned}
            x & = (x_1, x_2, \ldots, x_n)^T \\
            c & = (c_1, c_2, \ldots, c_n)^T \\
            b & = (b_1, b_2, \ldots, b_m)^T
        \end{aligned} \\
        A
        = \{a_{ij}\}_{i,j = 1}^{m,n}
        =
        \begin{pmatrix}
            a_{11} & a_{12} & \ldots & a_{1n} \\
            a_{21} & a_{22} & \ldots & a_{2n} \\
            \vdots & \vdots & \ddots & \vdots \\
            a_{m1} & a_{m2} & \ldots & a_{mn}
        \end{pmatrix}
        % = \begin{bmatrix}a_1 \\ a_2 \\ \vdots \\ a_n\end{bmatrix}
        % = [a^1, a^2, \ldots, a^n]
    \end{gathered}
\end{equation}

Мы в основном предполагаем что система совместна и неизбыточна.
Система избыточна если можно представить хотя бы одно из ее уравнений (неравенств) в виде линейной комбинации остальных.

Более развернутая форма:
\begin{equation}
    \begin{gathered}
        \max z = \max (c_1x_1 + c_2x_2 + \ldots + c_nx_n) \\
        \begin{pmatrix}
            a_{11} & a_{12} & \ldots & a_{1n} \\
            a_{21} & a_{22} & \ldots & a_{2n} \\
            \vdots & \vdots & \ddots & \vdots \\
            a_{m1} & a_{m2} & \ldots & a_{mn}
        \end{pmatrix}
        \begin{pmatrix}
            x_1 \\ x_2 \\ \vdots \\ x_n
        \end{pmatrix}
        \le
        \begin{pmatrix}
            b_1 \\ b_2 \\ \vdots \\ b_m
        \end{pmatrix}
        \\
        (x_1, x_2, \ldots, x_n)^T \ge 0
    \end{gathered}
\end{equation}

\newpage

\section{Двойственность}

\subsection{Двойственная задача для стандартной задачи максимизации}

В английском используются термины primal и dual.
На русском будем говорить прямая и двойственная.

Двойственная задача для стандартной задачи максимизации (\ref{eq:primal_max}):
\begin{equation} \label{dual_max}
    \begin{aligned}
        \min \overline z & = \min b^Ty \\
        A^Ty & \ge c \\
        y & \ge 0
    \end{aligned}
\end{equation}

В двойственной задаче у нас наоборот \(m\) переменных и \(n\) ограничений
\begin{equation}
    \begin{gathered}
        y = (y_1, y_2, \ldots, y_m)^T \\
        A^T =
        \begin{pmatrix}
            a_{11} & a_{21} & \ldots & a_{m1} \\
            a_{12} & a_{22} & \ldots & a_{m2} \\
            \vdots & \vdots & \ddots & \vdots \\
            a_{1n} & a_{2n} & \ldots & a_{mn}
        \end{pmatrix}
    \end{gathered}
\end{equation}

Более развернутая запись:
\begin{equation}
    \begin{gathered}
        \min \overline z = \min (b_1y_1 + b_2y_2 + \ldots + b_my_m) \\
        \begin{pmatrix}
            a_{11} & a_{21} & \ldots & a_{m1} \\
            a_{12} & a_{22} & \ldots & a_{m2} \\
            \vdots & \vdots & \ddots & \vdots \\
            a_{1n} & a_{2n} & \ldots & a_{mn}
        \end{pmatrix}
        \begin{pmatrix}
            y_1 \\ y_2 \\ \vdots \\ y_m
        \end{pmatrix}
        \ge
        \begin{pmatrix}
            c_1 \\ c_2 \\ \vdots \\ c_n
        \end{pmatrix}
        \\
        (y_1, y_2, \ldots, y_m)^T \ge 0
    \end{gathered}
\end{equation}

\subsection{Теоремы двойственности}

Теорема о слабой двойственности:
\begin{equation}
    c^T x \le b^T y
\end{equation}

Теорема о сильной двойственности говорит что если одна задача имеет решение то и вторая имеет решение и эти решения такие:
\begin{equation}
    c^T x^* = b^T y^*
\end{equation}

\(x*\) называется оптимальным решением, а значение целевой функции при нем называется значением задачи.

\subsection{Двойственность задачи ЛП в общем, стандартном и каноническом виде}

Прямая задача максимизации в общем виде:
\begin{equation}
    \begin{gathered}
        \max z = \max c^T x \\
        \begin{aligned}
            a_ix & \le b_i, \quad i = \overline{1, m_1} \\
            a_ix & = b_i, \quad i = \overline{m_1 + 1, m} \\
            x_j & \ge 0, \quad j = \overline{1, n_1} \\
            x_j & \in R, \quad j = \overline{n_1 + 1, n} \\
        \end{aligned}
    \end{gathered}
\end{equation}

Двойственная к ней:
\begin{equation}
    \begin{gathered}
    \min \overline z = \min b^T y \\
        \begin{aligned}
            (a^j)^Ty & \ge c_j, \quad j = \overline{1, n_1} \\
            (a^j)^Ty & = c_j, \quad j = \overline{n_1 + 1, n} \\
            y_i & \ge 0, \quad i = \overline{1, m_1} \\
            y_i & \in R, \quad i = \overline{m_1 + 1, m} \\
        \end{aligned}
    \end{gathered}
\end{equation}

Частный случай прямой задачи при котором она находится в стандартном виде означает что \(m_1 = m\) и \(n_1 = n\).
При таких значениях все ограничения являются неравенствами и у всех переменных есть ограничение на знак.
\begin{equation}
    \begin{gathered}
        \max z = \max c^T x \\
        \begin{aligned}
            a_ix & \le b_i, \quad i = \overline{1, m} \\
            x_j & \ge 0, \quad j = \overline{1, n} \\
        \end{aligned}
    \end{gathered}
\end{equation}

Двойственная задача к этой стандартной задаче будет тоже иметь все ограничения в виде неравенств и будет иметь ограничения на знак для переменных.
\begin{equation}
    \begin{gathered}
    \min \overline z = \min b^T y \\
        \begin{aligned}
            (a^j)^Ty & \ge c_j, \quad j = \overline{1, n} \\
            y_i & \ge 0, \quad i = \overline{1, m} \\
        \end{aligned}
    \end{gathered}
\end{equation}

Частный случай прямой задачи при котором она находится в каноническом виде означает что \(m_1 = 0\) и \(n_1 = n\).
При таких значениях все ограничения являются равенствами и у всех переменных есть ограничение на знак.
\begin{equation}
    \begin{gathered}
        \max z = \max c^T x \\
        \begin{aligned}
            a_ix & = b_i, \quad i = \overline{1, m} \\
            x_j & \ge 0, \quad j = \overline{1, n} \\
        \end{aligned}
    \end{gathered}
\end{equation}

Двойственная задача к этой канонической задаче будет иметь все ограничения в виде неравенств и не будет иметь ограничений на знак для переменных.
\begin{equation}
    \begin{gathered}
    \min \overline z = \min b^T y \\
        \begin{aligned}
            (a^j)^Ty & \ge c_j, \quad j = \overline{1, n} \\
            y_i & \in R, \quad i = \overline{1, m} \\
        \end{aligned}
    \end{gathered}
\end{equation}

\newpage

\section{Базисные решения}

Нас интересует отыскание неотрицательных решений системы линейных неравенств.

Можно показать что крайние точки выпуклого множества решений соответствуют так называемым базисным решениям.
(Они так называются потому что любую точку выпуклого множества можно представить в виде линейной комбинации его крайних точек.)

\subsection{Базисное решение СЛАУ}

Пусть
\begin{equation} \label{eq:basic_system}
    \begin{gathered}
        Ax = b \\
        A =
        \begin{bmatrix}
            B & \vline & N
        \end{bmatrix}
        , \quad
        x =
        \begin{bmatrix}
            x_B \\ x_N
        \end{bmatrix}
    \end{gathered}
\end{equation}

\(B\) --- невырожденная квадратная матрица.

Если положить
\begin{equation}
    \begin{gathered}
        x_B = B^{-1}b, \quad x_N = 0 \\
        x = 
        \begin{bmatrix}
            B^{-1}b \\ 0
        \end{bmatrix}
    \end{gathered}
\end{equation}
то получим решение системы (\ref{eq:basic_system}), называемое базисным решением.
Компоненты вектора \(x_B\) называются базисными переменными, а компоненты вектора \(x_N\) называются небазисными переменными.

Столбцы матрицы \(B\) называются базисными векторами, а ее \(m\) независимых столбцов образуют базис.

С другой стороны, если у нас есть решение \(x\), то оно называется базисным, если вектор-столбцы, соответствующие ненулевым компонентам \(x\), линейно независимы.
(Вектор столбцы из матрицы \(A\).)

\subsection{Неравенства}

Идея базисных решений оказывается полезной применительно для линейных неравенств.
Можно превратить их в равенства, добавив слабые переменные, а можно рассмотреть непосредственно сами неравенства.

\begin{equation}
    Ax \ge b, \quad x \ge 0
\end{equation}

Таким образом, в системе имеется \(m + n\) неравенств.
Если система совместна, она определяет непустое выпуклое множество.

Базисным допустимым решением является крайняя точка выпуклого множества, в которой по крайней мере \(n\) неравенств выполняются как равенства.
Поскольку оно допустимое, оно удовлетворяет и остальным \(m\) неравенствам.

\subsection{Определения}

Когда одна из базисных переменных равняется нулю, базисное решение называется вырожденным, и это соответствует ситуации когда в точке пересекается более чем \(n\) гиперплоскостей.

Поиск базисного решения это лишь поиск линейно независимой системы из \(m\) столбцов матрицы \(A\).

\textit{Утверждение:} Если у системы линейных уравнений существует решение, у нее существует и базисное решение.

\textit{Утверждение:} Если задача ЛП имеет допустимое решение, то она имеет и допустимое базисное решение.

\textit{Утверждение:} Если задача ЛП имеет оптимальное решение, то она имеет и оптимальное базисное решение.

В силу этого утверждения симплекс метод оперирует только с базисными решениями.

Небазисные допустимые решения являются внутренними точками множества допустимых решений.

\subsection{Геометрическая интерпертация слабых переменных}

Слабые переменные это \(s\).
Их количество равно \(m\) в случае перехода от стандартной задачи к канонической.
% Удобно не забывать что мы работаем с пространством переменных \(x\), и в него не входят \(s\).

Добавлением слабой переменной к каждому неравенству \(a_ix \le b_i\) мы превращаем систему неравенств в линейно независимую систему уравнений
\begin{equation}
    Ax + s = b
\end{equation}
% (В случае общей задачи Гаусс нам тоже дает подмножество уравнений которые линейно независимы и мы их тоже можем так рассмотреть.)

На каждом шаге мы разбиваем множество переменных
\(\begin{bmatrix}
    x \\ s
\end{bmatrix}\)
на базисные \(x_B\) и небазисные \(x_N\).
Небазисных переменных у нас \(n\) (размерность пространства решений).
Базисных переменных у нас соответственно \(m\).

Каждая слабая переменная \(s_i\) пропорциональна расстоянию от решения \(x\) до гиперплоскости, порождаемой соответствующим ограничением \(a_ix \le b_i\).
Приравнивание слабой переменной к нулю означает выполнение равенства, или, что то же самое, что \(x\) лежит в гиперплоскости этого ограничения.

Приравнивание к нулю \(n\) таких переменных \(s_i\) для их линейно независимых гиперплоскостей означает что \(x\) лежит во всех этих гиперплоскостях одновременно.
А пересечение \(n\) гиперплоскостей, вообще говоря, дает нам конкретную точку в пространстве.
Если эта точка допустимая, то она соответствует вершине выпуклого множества решений, и соответствует базисному решению.

Если же мы приравниваем к нулю одну из переменных \(x_i\), тогда хотя бы одна слабая \(s_j\) должна быть положительной, что значит \(x\) не лежит в гиперплоскости, порожденной ограничением \(a_jx \le b_j\).
В этом случае \(x\) вместо этого лежит в гиперплоскости, порожденной ограничением \(x_i \ge 0\).

То есть в любом случае если мы обнуляем \(n\) переменных из \(s\) и \(x\), точка будет лежать в \(n\) гиерплоскостях. А если при этом точка еще и допустимая, то она является вершиной множества.

\newpage


\section{Основы симплекс метода}

Это метод решения задачи ЛП.
Это численный метод решения задачи ЛП [Гейл].

Он истекает из графического метода решения и из факта что оптимальное решение будет находиться в краевой точке множества которое задано нашими ограничениями.
Краевая точка это такая точка множества которая не принадлежит ни одному отрезку множества.

Граничная точка может лежать на ребре а краевые это только вершины.

Симплекс метод --- алгоритм решения оптимизационной задачи линейного программирования путем перебора вершин выпуклого многогранника в многомерном пространстве.

Сущность метода: построение последовательности базисных решений, на которой монотонно убывает (возрастает) линейный функционал до ситуации, когда выполняются необходимые и достаточные условия локальной оптимальности [Википедия].

В работе Канторовича 1939 впервые были изложены принципы отрасли которую потом назвали линейным программированием.

\subsection{Принцип симплекс метода}

Принцип симплекс метода состоит в том что мы выбираем одну вершину многогранника и перемещаемся по ребрам в другие вершины в сторону увеличения функционала.

Когда такой переход невозможен считается что оптимальное значение найдено (либо его нет).

Симплекс метод можно поделить на две основные фазы:

\begin{enumerate}
    \item Нахождение исходной вершины множества допустимых решений,
    \item Последовательный переход от одной вершины к другой, ведущий к оптимизации значения целевой функции.
\end{enumerate}

Из-за того что в некоторых случаях нахождение исходной вершины тривиально, симплекс метод бывает однофазным и двухфазным.
То есть в тривиальном случае первую фазу опускаем.
Тривиальным случаем может быть когда 0 --- допустимое решение

\subsection{Симплекс таблицы}

Симплексная таблица (СТ) - основной элемент вычислительной процедуры симплекс метода.
Симплексная таблица представляет собой таблицу коэффициентов диагональной формы, построенной для канонической задачи максимизации [Кузютин].
Из-за того что она диагональная, она соответствует базисному решению рассматриваемой системы линейных уравнений.

\subsection{Допустимость}

Симплексная таблица называется прямо-допустимой если \(b \ge 0\).
Симплексная таблица называется двойственно допустимой если \(c \ge 0\).
Симплексная таблица называется оптимальной если она одновременно и прямо-допустимая и двойственно допустимая.
Оптимальная СТ соответствует оптимальному базисному решению.

\subsection{Симплекс алгоритм}

Алгоритм начинается с прямо-допустимой симплексной таблицы.

\begin{enumerate}
    \item
        На первом шаге мы выбираем ведущий столбец.
        Выбираем столбец с минимальным отрицательным \(c\).
        Если таких нет, то оптимальное решение найдено.
    \item
        На втором шаге выбираем ведущую строку из строк у которых элемент этого столбца положительный.
        Если положительных нет, задача ЛП не омеет оптимального решения.
        Выбираем строку с минимальным \(b/a\).
    \item
        На третьем шаге превращаем ведущий элемент в 1 и остальные элементы столбца в 0 эквивалентными преобразованиями.
        То есть мы проводим процедуру Гаусса по приведению таблицы к диагональному виду по новому базису.
\end{enumerate}

Вообще говоря выбор переменной с самой отрицательной \(c\) не гарантирует нам самую быструю сходимость и можно подобрать пример для которого алгоритм будет так обходить все вершины [Ху].

\newpage

\section{Матричное представление симплекс таблиц}

Симплекс таблица это способ решения задачи ЛП и способ записи решения симплекс метода.

\subsection{Исходная таблица}

Стандартную задачу максимизации можно переписать в каноническом виде таким образом:

\begin{equation} \label{eq:matrix_form_1_phaze}
    \begin{gathered}
        \begin{bmatrix}
            1 & -c^T & 0 \\
            0 & A & E_m
        \end{bmatrix}
        \begin{bmatrix}
            z \\ x \\ s
        \end{bmatrix}
        =
        \begin{bmatrix}
            0 \\ b
        \end{bmatrix}
        \\
        z \rightarrow \max, \quad x \ge 0, \quad s \ge 0
    \end{gathered}
\end{equation}

Это и есть симплекс таблица в матричной форме.

\begin{itemize}
    \item \(s \ge 0\) --- новые переменные, называющиеся слабыми, дополняющие старые таким образом, что неравенства переходят в равенства.
    \item \(z\) --- это переменная которую необходимо максимизировать, значение нашей целевой функции.
    \item
        Еще в википедии есть что \(b \ge 0\).
        Это как раз условие того что симплекс алгоритм должен начинаться с прямо-допустимой таблицы.
\end{itemize}

Более раскрытая форма для понимания:

\setcounter{MaxMatrixCols}{20}
\begin{equation} \label{eq:matrix_form_1_phaze_expanded}
    \begin{pmatrix}
        1 & \vline & -c_1 & -c_2 & \ldots & -c_n & \vline & 0 & 0 & \ldots & 0 \\
        \hline
        0 & \vline & a_{11} & a_{12} & \ldots & a_{1n} & \vline & 1 & 0 & \ldots & 0 \\
        0 & \vline & a_{21} & a_{22} & \ldots & a_{2n} & \vline & 0 & 1 & \ldots & 0 \\
        \vdots & \vline & \vdots & \vdots & \ddots & \vdots & \vline & \vdots & \vdots & \ddots & \vdots \\
        0 & \vline & a_{m1} & a_{m2} & \ldots & a_{mn} & \vline & 0 & 0 & \ldots & 1 \\
    \end{pmatrix}
    \begin{pmatrix}
        z \\ \hline x_1 \\ x_2 \\ \vdots \\ x_n \\ \hline s_1 \\ s_2 \\ \vdots \\ s_m
    \end{pmatrix}
    =
    \begin{pmatrix}
        0 \\ \hline b_1 \\ b_2 \\ \vdots \\ b_m
    \end{pmatrix}
\end{equation}

% Нужно написать так же матрично преобразования этих симплекс таблиц.
%
% Пусть матрица
% \begin{equation}
%     H = 
%     \begin{bmatrix}
%         A & \vline & E_m
%     \end{bmatrix}
% \end{equation}
%
% Мы хотим преобразовать эту марицу так, чтобы все переменные \(x\) стали базисными.
% Базисных переменных в любой момент времени у нас будет \(m\).
% Небазисных переменных у нас в любой момент времени будет \(n\).
%
% Базисные переменные образуют диагональную матрицу.
% И каждой базисной переменной соответствует свое уравнение.
% Базисная переменная принимает значение \(b\) потому что все небазисные переменные приравниваются к нулю, а базисные имеют нулевые коэффициенты в этой точке.

\subsection{Таблица с другим базисом}

Пусть матрица \(B\) состоит из столбцов матрицы \(
    \begin{bmatrix}
        A & \vline & E_m
    \end{bmatrix}
\), соответствующих базисным переменным.
Пусть также \(c_B\) это коэффициенты ЦФ, соответствующие этим переменным.

Тогда из исходной таблицы (\ref{eq:matrix_form_1_phaze}) можно получить таблицу для другого базиса:

\begin{equation} \label{eq:matrix_form_1_phaze_basis}
    \begin{bmatrix}
        1 & c_B^TB^{-1}A-c^T & c_B^TB^{-1} \\
        0 & B^{-1}A & B^{-1}
    \end{bmatrix}
    \begin{bmatrix}
        z \\ x \\ s
    \end{bmatrix}
    =
    \begin{bmatrix}
        c_B^TB^{-1}b \\ B^{-1}b
    \end{bmatrix}
\end{equation}

Этот переход эквивалентен умножению (\ref{eq:matrix_form_1_phaze}) слева на
\begin{equation}
    \begin{bmatrix}
        1 & c_B^TB^{-1} \\
        0 & B^{-1}
    \end{bmatrix}
\end{equation}

\subsection{Базисные переменные}

Что вообще из себя представляет матрица \(B^{-1}A\)?

Пусть переменная \(x_j\) находится в базисе на месте \(i\).
Тогда столбец матрицы \(B\) с индексом \(i\) равен столбцу \(a^j\):
\begin{equation}
    \begin{gathered}
        B = \begin{bmatrix}
            B^1 & \ldots & a^j & \ldots & B^m
        \end{bmatrix} \\
        B^i = a^j
    \end{gathered}
\end{equation}

Пусть матрица \(B^{-1}\) выглядит так:
\begin{equation}
    B^{-1} =
    \{\hat b_{ij}\}_{i,j=1}^{m,m}
    =
    \begin{bmatrix}
        \hat b_1 &
        \hat b_2 &
        \ldots &
        \hat b_m
    \end{bmatrix}
    =
    \begin{bmatrix}
        \hat b^1 &
        \hat b^2 &
        \ldots &
        \hat b^m
    \end{bmatrix}
\end{equation}

Матрица \(B^{-1}\) при умножении на столбец \(a^j\) превращает его в единичный.
Единица в нем стоит на месте \(i\):
\begin{equation}
    B^{-1}a^j = \begin{bmatrix}
        0 & \ldots & 1 & \ldots & 0
    \end{bmatrix} ^T
\end{equation}
\begin{equation}
    \begin{gathered}
        \hat b_i a^j = 1, \quad i = j \\
        \hat b_i a^j = 0, \quad i \ne j
    \end{gathered}
\end{equation}

Тогда \(B^{-1}A\) можно записать так:
\begin{equation}
    B^{-1}A =
    \begin{pmatrix}
        \hat b_1a^1 & \ldots & \hat b_1a^j & \ldots & \hat b_1a^n \\
        \vdots & & \vdots & & \vdots \\
        \hat b_ia^1 & \ldots & \hat b_ia^j & \ldots & \hat b_ia^n \\
        \vdots & & \vdots & & \vdots \\
        \hat b_ma^1 & \ldots & \hat b_ma^j & \ldots & \hat b_ma^n \\
    \end{pmatrix}
    =
    \begin{pmatrix}
        \hat b_1a^1 & \ldots & 0 & \ldots & \hat b_1a^n \\
        \vdots & & \vdots & & \vdots \\
        \hat b_ia^1 & \ldots & 1 & \ldots & \hat b_ia^n \\
        \vdots & & \vdots & & \vdots \\
        \hat b_ma^1 & \ldots & 0 & \ldots & \hat b_ma^n \\
    \end{pmatrix}
\end{equation}

Матрица \(B^{-1}\) при умножении на матрицу \(A\) превращает все столбцы соответствующие базисным переменным в единичные.
Очевидно, то же самое будет происходить и для переменных \(s\).
Столбцы, которые не соответствуют базисным переменным будут иметь всякие разные числа.

Таким образом эта операция, то есть умножение на \(B^{-1}\) эквивалентна использованию метода Гаусса (или же эквивалентных преобразований) в симплекс алгоритме.
А именно, она нам приводит матрицу \(
    \begin{bmatrix}
        A & \vline & E_m
    \end{bmatrix}
\) к трапецевидной (или диагональной) форме по базисным переменным.

\subsection{Таблица в развернутом виде}

Перепишем таблицу (\ref{eq:matrix_form_1_phaze_basis}) в развернутом виде:

\begin{equation} \label{eq:c_b_hat}
    \hat c_B =
    c_B^TB^{-1} =
    \begin{pmatrix}
        c_B^T \hat b^1 &
        c_B^T \hat b^2 &
        \ldots &
        c_B^T \hat b^m
    \end{pmatrix}
\end{equation}
\begin{equation} \label{eq:matrix_form_1_phaze_basis_expanded}
    \begin{pmatrix}
        1 & \vline & c_B^TB^{-1} a^1-c_1 & \ldots & c_B^TB^{-1} a^n-c_n & \vline & \hat c_B_1 & \ldots & \hat c_B_m \\
        \hline
        0 & \vline & \hat b_1a^1 & \ldots & \hat b_1a^n & \vline & \hat b_{11} & \ldots & \hat b_{1m} \\
        0 & \vline & \hat b_2a^1 & \ldots & \hat b_2a^n & \vline & \hat b_{21} & \ldots & \hat b_{2m} \\
        \vdots & \vline & \vdots & \ddots & \vdots & \vline & \vdots & \ddots & \vdots \\
        0 & \vline & \hat b_ma^1 & \ldots & \hat b_ma^n & \vline & \hat b_{m1} & \ldots & \hat b_{mm} \\
    \end{pmatrix}
    \begin{pmatrix}
        z \\ \hline x_1 \\ x_2 \\ \vdots \\ x_n \\ \hline s_1 \\ s_2 \\ \vdots \\ s_m
    \end{pmatrix}
    =
    \begin{pmatrix}
        \hat c_B b \\ \hline \hat b_1 b \\ \hat b_2 b \\ \vdots \\ \hat b_m b
    \end{pmatrix}
\end{equation}

\subsection{Верхняя строка таблицы}

Строка \(
\begin{pmatrix}
    c_B^TB^{-1} a^1-c_1 & \ldots & c_B^TB^{-1} a^n-c_n & \vline & \hat c_B_1 & \ldots & \hat c_B_m
\end{pmatrix}
\) будет иметь нули на позициях соответствующим базисным переменным.

Мы уже выяснили что \(B^{-1}a^i\) превращает столбец в единичный если переменная \(x_i\) является базисной. Поэтому умножение этого произведения на \(c_B^T\) слева дает нам коэффициент ЦФ этой базисной переменной \(x_i\), и дальнейшее вычитание опять этого коэффициента обнуляет соответствующий элемент в строке.

Для переменных \(s\) всё еще проще, оно аналогично, только вместо матрицы \(A\) у нас матрица \(E_m\), а изначальные коэффициенты уже равны нулю, поэтому вычитание опущено.

% Пусть \(x'\) базисные а \(x''\) небазисные.
%
% \(Ax + s = b \Leftrightarrow Bx' + Dx'' = b \Leftrightarrow x' + B^{-1}Dx'' = B^{-1}b\).
%
% \(z - c^Tx = 0\).
%
% \(c_B^TB^{-1}Ax + c_B^TB^{-1}s = c_B^TB^{-1}b\).

Таким образом, на каждом шаге \(m\) столбцов таблицы (\ref{eq:matrix_form_1_phaze_basis_expanded}), соответствующие базисным переменным являются линейно независимыми единичными столбцами.

\subsection{Условия оптимальности и допустимости}

Когда в верхней строчке в таблице (\ref{eq:matrix_form_1_phaze_basis_expanded}) нет отрицательных элементов, оптимальное решение найдено и записано в правой части уравнения.
Если отрицательные элементы есть то ищем среди них минимальный и соответствующая переменная \(x_i\) из небазисных будет вводимой:
\begin{equation} \label{eq:matrix_form_optimality}
    \min_i (c_B^TB^{-1} a^i-c_i)
\end{equation}

% Эта таблица прямо-допустима когда в правой части нет отрицательных элементов не считая первого.
% В тахе есть другое понятие.
Чтобы выбрать какую переменную убрать из базиса, мы выбираем элемент среди базисных.
Если \(x_i\) это вводимая переменная, то исключаемой будет \(x_j\) та у которой
\begin{equation} \label{eq:matrix_form_feasibility}
    \min_j \frac{\hat b_jb}{\hat b_ja^i}
\end{equation}

Для переменных \(s\) эти выражения аналогично составляются.

\subsection{Модифицированный симплекс метод}

Алгоритм с матричной формой записи, называется модифицированным симплекс алгоритмом:

\begin{enumerate}
    \item Строим исходную таблицу (\ref{eq:matrix_form_1_phaze}).
    \item Строим матрицу столбцов текущих базисных переменных \(B\) и находим обратную к ней \(B^{-1}\).
    \item С помощью \(B^{-1}\) получаем текущую таблицу (\ref{eq:matrix_form_1_phaze_basis}).
    \item Проверяем условия оптимальности (\ref{eq:matrix_form_optimality}) и допустимости (\ref{eq:matrix_form_feasibility}), находим новый базис и возвращаемся к шагу 2.
\end{enumerate}

Можно оптимизировать этот алгоритм, если не пересчитывать на каждом шаге обратную матрицу заново, ведь в исходной меняется лишь один столбец. В этом как то может помочь LU разложение.

Но если не пересчитывать обратную полностью будет накапливаться вычислительная ошибка.
Мы можем отслеживать как она накапливается, наблюдая за элементами, которые мы знаем должны равняться нулю. Когда они становятся больше некоторого заданного \(\epsilon\), мы пересчитываем \(B^{-1}\) по новой.

\subsection{Решение двойственной задачи}

Пусть \(B\) составлена по оптимальному базису.

Тогда решение двойственной задачи равно:
\begin{equation}
    y* = c_B^TB^{-1}
\end{equation}

\section{Алгоритм решения с ограниченными переменными}

Это когда \(x \ge l\) вместо \(x \ge 0\) или \(x \ge l\).

Если ограничение снизу то получается простая замена переменых.

Если ограничение сверху, то замена переменных не сработает.
Можно конечно расширить матрицу ограничений и включить их туда, но можно поступить умнее.

Мы можем проверять эти ограничения на шаге проверки допустимости (\ref{eq:matrix_form_feasibility}).

Добавляются еще две проверки, и переменная вводится в решение бла бла бла...

\newpage

\section{Пример}

\begin{equation}
    A = \begin{pmatrix}
        -3 & 4 \\
        0 & 1 \\
        3 & 2 \\
        3 & -1
    \end{pmatrix}
    , \quad
    b =
    \begin{pmatrix}
        12 \\
        6 \\
        42 \\
        33
    \end{pmatrix}
    , \quad 
    x = \begin{pmatrix}
        x_1 \\ x_2
    \end{pmatrix}
    , \quad
    c = \begin{pmatrix}
        2 \\ 3
    \end{pmatrix}
\end{equation}

\begin{equation}
    \begin{aligned}
        \max z & = \max c^Tx \\
        Ax & \le b \\
        x & \ge 0
    \end{aligned}
\end{equation}

\newpage

\section{Источники}

\begin{enumerate}
    \item Конспект-книжка Кузютина Дениса Вячеславовича
    \item Гейл
    \item Таха
    \item Ашманов
    \item Ху
    \item Схрейвер
    \item Васильев
    \item Интернет
\end{enumerate}

\end{document}
